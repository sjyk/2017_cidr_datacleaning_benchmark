\section{Background}

\begin{itemize}
    \item Walk through an example ETL/Data Cleaning pipeline, how do you evaluate this, how do you test for issues, how do you test competing algorithms.
    \item Describe the current issues and challenges in evaluating data cleaning.
    \item Justify some of the key design decisions w.r.t language and tool for this specific project.
\end{itemize}


Despite all the different ways that data can be wrong, the they can be characterized by a concise set of declarative operations.  We base much of our approach for error synthesis and generation based on these operations.  LIST THEM.

Give examples for an operation that are seemingly different, but all boil down to the same operator.

Our insight is that the parameters to these operators are precisely what the domain experts can provide.

Conversely, this simple algebra offers a concise way of {\it describing} the errors present in a dataset.  It is difficult to summarize data~\cite{bhardwaj2015collaborative}, and current approaches heavily utilize visualization or statistical summarization techniques, which can be difficult to understand.  In contrast, presenting a sequence of data error operations, along with their parameters, as a summary of a dirty dataset, would be akin to presenting a SQL query that summarizes a transformed dataset.  In fact, view synthesis work exists.


